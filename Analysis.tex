% Created 2020-07-05 dom 20:29
% Intended LaTeX compiler: pdflatex
\documentclass[11pt]{article}
\usepackage[utf8]{inputenc}
\usepackage[T1]{fontenc}
\usepackage{graphicx}
\usepackage{grffile}
\usepackage{longtable}
\usepackage{wrapfig}
\usepackage{rotating}
\usepackage[normalem]{ulem}
\usepackage{amsmath}
\usepackage{textcomp}
\usepackage{amssymb}
\usepackage{capt-of}
\usepackage{hyperref}
\author{Gabriel Petrini da Silveira}
\date{\today}
\title{}
\hypersetup{
 pdfauthor={Gabriel Petrini da Silveira},
 pdftitle={},
 pdfkeywords={},
 pdfsubject={},
 pdfcreator={Emacs 26.3 (Org mode 9.3.7)}, 
 pdflang={English}}
\begin{document}

\tableofcontents

\section{Introdução}
\label{sec:orge111781}

Este arquivo contém os códigos e a análise dos resultados da simulação dos experimentos. 
Por padrão, os efeitos serão comparados em relação ao baseline.
Os experimentos são de duas naturezas: (i) política econômica e (ii) comportamental.
Os primeiros tratam das mudanças nos parâmetros associados ao governo enquanto os segundos dizem respeito à dispersão dos parâmetros da função de consumo dos consumidores.
Abaixo uma tabela detalhando cada um desses experimentos e associando ao respectivo arquivo de configuração do LSD.

\section{Configurações globais}
\label{sec:orgbd55e52}




\begin{verbatim}
import pandas as pd
import numpy as np
import matplotlib.pyplot as plt

discart = 0.1 # Serão desconsiderados os 10% primeiros períodos da simulação
teste = pd.read_csv('./Data/Teste.txt', sep = ',')
return teste.head()
\end{verbatim}



\section{Política econômica}
\label{sec:org79d8b42}

\begin{verbatim}
teste = pd.DataFrame('Teste.txt', sep = ',')
teste.head()
\end{verbatim}

\section{Comportamental}
\label{sec:org39f9ceb}

\begin{verbatim}
head(mtcars)
\end{verbatim}
\end{document}